%!TEX root = ../thesis.tex
%*******************************************************************************
%****************************** Second Chapter *********************************
%*******************************************************************************

\chapter{Single molecule somatic mutation detection}

\ifpdf
    \graphicspath{{Chapter2/Figs/Raster/}{Chapter2/Figs/PDF/}{Chapter2/Figs/}}
\else
    \graphicspath{{Chapter2/Figs/Vector/}{Chapter2/Figs/}}
\fi


% Uncomment this line, when you have siunitx package loaded.
%The SI Units for dynamic viscosity is \si{\newton\second\per\metre\squared}.

%% BC-1, HT-115 cell lines
%% blood: SBS1, SBS5 + SBSX: 
%% emily mitchell's paper
%% mia's paper
%% single dominant somatic mutational processes drive the generation of the most recently acquired somatic mutations
%% an individual/single substitution/difference between the read and reference genome cannot be determined to be a sequencing error or a true mutation that reflects the difference between the sample and the reference
%% if the circular consensus sequence base from Pacific Biosciences are sufficiently accurate, the differences between the read and the reference should be a reflection of biology and not the reflection of library errors, sequencing errors, systematic errors.   
%% if single molecule somatic single-base substitutions, in aggregate, should be consistent with the expected mutational patterns from the sample if the mutations are correctly called.
%% sources of library errors, sonication, oxidative DNA damage
%% 
%% single moelcule somatic single base substitution detection across all  eukaryotic species, potentially sequenced with PacBio CCS sequencing method
%% even if soruces of errors introduced upstream of sequencing can be identified, it is hard to deal with them
%% systematic errors resulting from alignment errors and

%% Duplex sequencing
%% Nanorate-sequencing
%% Darwin Tree of Life project
%% Importance of somatic mutations: driver mutations, lineage trace development, time the emergence of driver mutations

\section{Introduction}

Mutations can occur in cells at all stages of life and in all tissues and clonality and classification is dependent on tissue-type, size of the genetic modification and time of the mutation. Germline mutations are mutations the genetic polymorphisms that children inherits from their parents. Child might have mutations that are private from their parents as a result of de novo mutations occuring during cell division and mutagenesis during meiotic recombniation (discussed in Chatper 5, reference). Somatic mutations are mutations that occur post-fertilisation. If somatic mutation occurs during first-cell division of embryonic development, for example, somatic mutation will be present on all the daugther cells originating from the embroynic cell with the somatic mutation. Somatic mutations can be differently classified depending on ther size and their orientation. Most somatic mutations are benign, but some somatic mutations confer proliferative advantage to the cell and these somatic mutations are referred to as driver mutations. Somatic muations can be substitution, small insertions and deletions (<50bp) or large changes to genomes (>50bp), also known as structural variations. The advent of next-generation sequencing has enabled us to detect more complex forms of genomic rearrangemetns such as chrmothripsis, chromoplexy, break-induced replication, fork stalling and template switching, etc and the elucidation of DNA damage and repair pathways that generate these complex rearrangements. Somatic mutation detection, hence, is often the first-step towards characterising the cancer genome. Thanks to decrease in cost for whole-genome sequencing, thousands of cancer genomes have been sequenced, tissue-specific driver mutations and somatic mutational processes have been uncovered. However, due to the technical limitations of the NGS platform limit the length, clonality, diversity and resolution of the somatic mutations that can be detected. The base quality score, measuring the probability that the base read out is a sequnencing error and not a true representasion of the template molecule, ranges from 0 to 40, 1 error per 10,000 bp, for Illumina short reads and as a result, somatic mutation caller such as Mutect [reference], Strelka2 [reference] requires matched tumour and normal sequencing to distinguish germline mutations from somatic mutations and cannot distinguish low allelic fraction somatic mutations from sequencing errors. In addition, repetitive sequences account for 50\% of the human genome and repeat length is often greater than the short read length. Moreover, there isn't one type of repeat, multiple types of repeats are present in the human genome (tandam repeat expansions, retrotransposons like ALUs and SINEs, non-repetitive unique sequence copies such as segmental duplications, alpha-satellites in centromeric regions and centromeric repetitive sequences) and these repetitive structures are responsible for somatic mutagenesis both small and big. If the repeat length is greater than the read length, read aligners that determines the position of read relative to the reference genome cannot ascertain the position of the read relative to the reference genome and these reads without fixed locations cannot be used to call somatic mutations. To determine the location of the read relative to the reference genome with high-confidence, the repetitive sequences within the read must be flanked by unique sequences not exisiting elsewhere in the reference genome. In addition, somation detection isn't a solved problem. Different somatic mutation callers with different strategies for somatic mutation calling have different sensitivity and specificity and often the consensus of different somatic mutation caller provides the most accurate somatic mutation calls. In addition, if tissue in question doesn't have sufficent DNA and requires PCR amplification, somatic mutation detection might also be confounded with PCR amplifiaction bias.


The technical limiations of next-generation sequencing platform have impededed our understanding of the transition from normal cells to neoplastic cells as normal cell development could not be directly monitored through the detection of low variant allele fraction (VAF) mutations. To overcome the difficulties associated with somatic mutation detection with Illumina reads too approaches have been mainy developed. One approach aims to increase the clonality of the mutant DNA that might exist in a single cell or a group of cells and the other approach aims to increase the base accuracy of the reads through clever upstream changes in the Illumina library preparation protocol. Single-cell whole-genome amplification, single-cell colony expansion and laser-capture microdissection (LCM) and sequencing belongs to the former camp and duplex sequening based methods and variations thereof belong to the latter camp. In the former approach, the clonality of the mutant DNA is increased above the limit of detection threshold. Single-cell whole-genome amplification and sequencing, unfortunately, suffers from PCR amplification bias and allelic bias and hence, doesn't provide the most accurate somatic mutation calls and single-cell clone expansion and sequencing offers the most accurate somatic mutation call among the approaches that elevates the VAF of mutant DNA above the limit of detection threshold. Duplex sequencing approach, inspired by the approach, that allows the use of DNA as a stable structure across millions of years uses the redundancy of complementary information between the forward and reverse strand of DNA double helix to achieve high base accuracy. In brief, DNA extract is fragmented through sonication,  unique molecular identifier (UMI), consisting of 8 to 12 nucleotides (nt) is attached to a double-stranded DNA through blunt-end/overhang ligation with Illumina adapters, double-stranded DNA is separated into single-stranded DNA, which undergoes PCR amplification to produce multiple copies of the samme molecule. The concentration of input DNA for PCR amplification needs to be controlled to enable optimal selection and amplification of both strands of the double-stranded DNA. Illumina library is sequenced. Thereafter, reads are grouped according to their UMI and read belonging to the same UMI are used to generate double-strand consensus (duplex) read taking advantage of the complementary information between the forward and reverse strand. During PCR amplification, DNA polymerase might incorrectly replicate the template, but the incorrectly introduced base will be present only in one of the single-strand read or fraction of the single-strand reads and will not have a complementary base compared to the reverse complementary strand. Except for PCR-jackpot errors that occurs at the very first-cycle of PCR amplification and that impacts all the subsequent copies of the single-strand, duplex read should only be affecteted by library errors introduced upstream of PCR amplifiaction and sequencing. Duplex sequencing, theoretically, promises base accuracy of 1 x 10-9, but in practice duplex reads have a base accuracy of 1 x 10-6. 

Nanorate sequencing improves the duplex library protocol to achieve the theoretical base accuracy of duplex reads. In contrast to the original duplex library protocol that repairs DNA molecules with DNA damage using end-repair enzymes, nick translation, and gap-filling enzymes, nanorate sequencing protocol prevents library preparation from DNA molecules with DNA damage. Instead of DNA fragmentation through sonication, nanorate sequencing protocol uses restriction enzymes that generates blunt-ends and through the addition of dideoxy nucleotides prevents DNA synthesis during nick translation. PacBio CCS sequencing shares many simliar features as duplex sequencing and we hypothesized that CCS reads might as accuraet or more accurate than duplex reads as CCS sequencing does not suffer from PCR amplifiaction and might have sufficient base accuracy to enable single molecule somatic mutation detection.


There are two types of third-generation sequencing: one from Pacific Biosciences and one from Oxford Nanopore Technologies and both companies attempt to sequence single molecule of DNA, in contrast to the sequencing by synthesis approach. These approaches, previsouly, had an error rate ranging from 20\% to 40\% depending on the library chemistry and the base caller version. PacBio introduced circular consensus sequencing in 2010, but circular consensus sequencing could not be adopted for mass-adoption as DNA polymerase for SMRT sequencing didn't have sufficient processivitiy to read long read-of-insert multiple times. Instead, PacBio offered continuous long read (CLR) sequencing to its customers which maximized for read length instead of average read accuracy. CLR reads typically have 10-15\% error rate, but is free from PCR amplification, the errors are thought to be randomly introduced and CLR reads have read length that is 100-fold longer than that from short reads. CLR reads, hence, was adopted for de novo assembly of complex genomes that could not be assembled with short reads and for structural variation detection. The longer read length enables the read alignment software to confidently assign the location of the reads relative to the reference genome as unique sequences are flanking repetitive sequences. Germline structural variation detection with long reads doubles the average number of structural variations discovered per genome compared to that from short reads and improves the diagnostic yield of rare genetic disease detection from 30\% to 80\%. The lower base accuracy and cost of SMRT sequencing, however, limited the wider adoption of PacBio SMRT sequencing except for one-off de novo assembly projects and clinical sequencing of patients with rare genetic diseases. PacBio, however, successfully engineered DNA polymerases with increased processivitiy and was further able to improve their circular consensus sequencing method such that read-of-insert with average read length of 10kb to 20kb can be read multiple times and because the errors are introduced randomly to each single-strand sequence templates, consensus sequence algorithms can take advantage of the complementary nature of double-stranded DNA to produce circular consensus sequences with average read accuracy greater than Q20.  

Based on our understanding of duplex sequencing and the improved nanorate sequencing protocol, we hypothesized that PacBio CCS reads might have sufficient base accuracy for single molecule somatic mutation detection as it shares many of the characteristics as duplex sequencing, but is free from PCR-jackpot errors that occurs in the earliest stages of PCR amplification for duplex sequencing protocols. PacBio CCS base quality score ranges from Q1 to Q93, representing error rate from 1 to 1 in 5 billion bases. If the base quality score estimates are correct, single molecule somatic mutation detection should be possible in human samples as the human genome somatic mutation rate is 1 to 2 somatic mutations per human genome per 1-4 weeks, which is equivalent to Q70 and the somatic mutation rate is thought to be consistent throughout a person's life time. An individual has a higher somatic mutation rate than that of a normal person, if an individual has defective mutations in enzymes associated with DNA damage and repair process. If single molecule somatic mutation detection is successful from PacBio CCS reads, we aimed to detect and characterise the somatic mutations and somatic mutational processes in 66,000 eukaryotic species from the Darwin Tree of Life project, which aims to sequence and assemble all the eukaryotic species in Britain and Ireland, providing an unparalleld opportunity to examine the somatic mutations and assocaited somatic mutational processes of non-human samples across the Tree of Life (disscussed in Chapter 3). 

In addition, to our greater understanding of the role of somatic mutations in tumour evolution, we also have a better understanding of the biological processes that generates these somatic mutations.

%%  Somatic mutation generation is 

We need to define the problem first. In contrast to germline mutation detection where the mutation caller attempts to detect mutations that is homozygous or heterozygous, which exist as 100\% variant allele fraction and 50\% variant allele fraction, respectively, somatic mutations caller aims to detect somatic mutations that might be present in a single cell to a somatic mutations that might be present in all of the cancer cells and take into account tumour purity into the calculation. Somatic mutation callers, hence, often require a matched-tumour normal sequencing to distinguish germline mutations from somatic mutations and to calculate tumour purity??. In addition, because of techical limitations of short-read sequencing, low frequency somatic mutation with variant allele fraction below 0.1-1\% often cannot be detected. 


To assess whether single molecule somatic mutation detection is possible with CCS reads, we sequenced three positive control samples (BC-1, HT-115, PD48473b) and one negative control sample (PD47269d, Table 1). The three positive control samples have a single somatic mutational process that is responsible for the generation of almost all the recently acquired somatic mutations in that sample. In contrast to the positive control sample, the cord blood sample should not have great number of somatic mutations and as a result, single-base substitutions detected from the negative control sample will be representative of the CCS error profile. In addition, the use of samples with single somatic mutational processes has the added benefit that these samples has been characterised in-depth through single-cell expansion and clone sequencing and we have determined the mutational probability of each substitution type in each trinucleotide sequence context. We, hence, are aware of the mutational pattern expected from the sample and can find the parameters that allows us to find mutational pattern from our positive control samples that is more consistent with what is expected from the sample. In addition, mutational signature analysis allows us to determine the number of mutations attributable to the correct biological process responsible for generating that somatic mutation and number of mutations attributable to false positive substitutions.  



%% somatic single base substitutions
%% small indels
%% structural variations > 50 bp
%% chromosomal translocations
%% tumour suppressor genes
%% technical limitations of illumina sequencing
%% normal sequencing: to understand the transformation of normal cells to neoplastic cells
%% DToL project
%% availability of reference genomes from diverse eukarytoic species 
%% reference genome construction: CCS sequencing, linked reads, CLR, binano scaffolding



\section{Materials \& Methods}
\subsection{Samples}
\subsection{Mutational signatures}
\subsection{Pacific Biosciences Circular Consensus Sequencing}
%% indel errors
%% homopolymers
%% continuous long read sequencing

%% error profile have not been characterised

\subsection{Single molecule somatic mutation calling}
\subsection{Methods for single molecule somatic mutation calling}
\subsection{Mutation calling}
\subsection{Hard filters}
\subsection{Haplotype Phasing}
\section{Benchmarks}
\subsection{Sensitivity and Specificity, F1-statistics}
\subsection{Receiver-operating characteristics}
\section{Results}
\subsection{}
\subsection{}
\section{Discussion}
\subsection{Liquid Biopsy}
\subsection{False positive substitutions}
\subsection{DeepConsensus}
\subsection{Environmental mutagenesis}
\subsection{Single molecule structural variation detection}
\subsection{Single molecule somatic mutation detection in non-human species}

