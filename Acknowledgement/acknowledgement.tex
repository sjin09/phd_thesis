% ************************** Thesis Acknowledgements **************************

\begin{acknowledgements}      


I would like to thank Professor Jeong-Sun Seo giving me a complete novice in sequence analysis to participate in the Korean Genome Project where I had the immense opportunity to assemble and analyse human genomes with the latest sequencing and mapping technologies.

I would also like to thank Wellcome Sanger Institute and the University of Cambridge in accepting my applications. It was heart-warming to return to University of Cambridge after my national service and I am always in awe when I realise that I am walk the same road and breath the same air as Charles Darwin, James Watson, Francis Crick, Ronald Fisher, Shankar Balasubramanian, and David Klenerman,
 (Heraclitus thinkers would disagree with this assertion), the giants that preceded modern genetics and genomics and that established the foundations for modern genetics and genomics. As Newton would say “if I have seen further [than others], it is by standing on the shoulders of giants”.

My high school friends: Jisoo Kim, Anuran Makur, Gaurav Kankanhali, Jinseok Lee
My Imperial College friends: Jongseok Ahn, Yunsung Na, William Gao, Rebecca Yu


This work would not have been possible without all the preceding work to bring genomics research to light.  

I would like to especially thank Anny King at Churchill College for her continual support and unending warmth. I would not have been able to complete my Mphil in Computational Biology without her support. Once I described that hugging Anny is like being enwrapped in the most soft and luxurious duvet and when I meet Anny, I cannot help but feel that there is light still for humanity. 

Great tutelage 
To not meet your heroes.

If I was not given the opportunity, I would not have had the chance to immerse myself in sequencing studies. I would like to thank my supervisor Peter Campbell for providing an original question: “what is the mutational process in non-human species” to answer. A scientist is often does not encounter an original question and often is challenged to come up with an original question and the question is often a derivative of other questions that was previously asked. I have nothing but gratitude for my two supervisors: Peter Campbell and Richard Durbin. They were both extremely generous with their time, provided me with sufficient independence to tackle the problem even when the problem could have been solved easily and more efficiently by someone else and when I struggled, they I knew that they would be able to help me. Despite Peter’s popularity, I cannot help but think that that it was interesting how other PhD students in year was not interested in application of long reads and that other people had no interest in somatic mutations. I am agnostic to Christianity, but sometimes I must say. I believe that “heaven helps those who help themselves” is appropriate for some circumstances. I would also to thank Peter and Richard for having the continued belief in me that I would be able to solve the problem in hand.  

I sometimes think that Peter and Richard has more belief in me than I had believed in myself and in turn their belief in me motivated me give my best so that I don’t disappoint them. 

The questions and answers that followed still astounds me. 

Peter used the analogy of killing the question first
If I were to become an academic in the future, I wish that I could emulate Peter and Richard’s style of supervision.
Time to tackle the question, courage to take the problem,
To elevate the way with which I answer the question
Convince me to arrive at better questions and better answers
Gently and subtly
To attack the question, to spot the vector with which to attack the question. (Richard Hamming)
To be unafraid to ask original questions
To revisit questions that could not be asked with past technologies.
To have the grand-arching theme under which all the questions are asked and answered.
Freedom to make mistakes
A more competent person, student or a post-doc to complete the work. 

Juan Park, Sulki Park, 


Child-like curiosity to new data and ways to analyse the data. The breadth and depth of their knowledge. If I could emulate them to obtain 1/100th of their knowledge, that would be life well spent. 

How systematic their thinking is. (the framework with which they tackle the question)

Patience to listen 

To provide the right connections

I would also like to thank the cgp-lab, long-read sequencing team and the Darwin Tree of Life project for DNA extraction, library preparation and sequencing. I would not been able to complete the project without their effort and the compute infrastructure. 


Raheleh Rahbari 

I would also like to thank my friends and colleagues at the Wellcome Sanger Institute (Thomas Oliver, Sigurgeir Olafsson, Lori Kregar, Mike Spencer Chapman, Emily Mitchell, Kenichi Yoshida, Hyunchul Jung, Haerin Jang, Chloe Pacyna, Rashesh Sanghvi, Yichen Wang, Jun Sung Park, Haynes Heaton) for their generosity and stimulating discussions and for making the Wellcome Sanger Institute a more exciting place. Special thanks are also given to Mike whom I enjoyed regular runs during the afternoon, and I was able to compete in multiple Cambridge half marathons as a result and a full marathon awaits me. In addition, Haynes has been instrumental for my mental well-being during the COVID-19 pandemic and introducing rust to me and helping with himut development during our regular pair programming sessions. 

I would also like to thank my high school friends Anuran Makur and Gaurav Kankanhalli who have already completed their PhD degrees and are already tenured-tracked professors and Jinseok Lee who is pursuing his PhD degree in Computational Biology at University of Northern Carolina for the stimulating discussions and support throughout the years. 

I would finally like to thank my family members for their continued support and warmth. 

I will dearly miss my times and conversations at the University of Cambridge and Wellcome Sanger Institute.



\end{acknowledgements}
