% ************************** Thesis Acknowledgements **************************

\begin{acknowledgements}      

\textit{"You can't connect the dots looking forward; you can only connect them looking backwards. So you have to trust that the dots will somehow connect in your future. You have to trust in something - your gut, destiny, life, karma, whatever."} 
\begin{flushright} [Steve Jobs' 2005 Stanford Commencement Address] \end{flushright}

I first encountered SMRT sequencing as a bioinformatics scientist at Macrogen in 2015, and the potential of single molecule sequencing was immediately apparent to me. Since then, much of this promise has been realized, while some opportunities remain untapped. At the time, the number of algorithms that could analyze and provide meaningful results from reads with a high error rate was very limited. The availability of continuous long reads, with an average read length (>10kb) longer than the common repeats in the human genome, and the capability to perform \textit{de novo} assembly of multi-megabase contigs and detect structural variations at nucleotide-resolution, started to generate excitement among scientists who were frustrated with the limitations of short-read sequencing. The simultaneous development of the 3D-DNA Hi-C scaffolding algorithm, which can correct assembly errors and order and orient contigs into chromosome-length scaffolds, completely transformed the time and cost required to sequence and assemble high-quality reference genomes.
 
I was particularly thrilled to explore uncharted biological phenomena using SMRT sequencing as a PhD candidate at the Wellcome Sanger Institute. Peter inspired me to evaluate the potential of CCS reads for somatic mutation detection across the Tree of Life and to design a somatic mutation detection algorithm that would be applicable to various tissues, species, and clonalities. At that time, I was the only PhD student passionate about pursuing this project, and I still cannot believe the incredible stroke of luck I experienced. Both Peter and Richard were exceptionally generous with their time, and what I appreciated most was their patience in allowing me to make and learn from my mistakes, enabling me to truly take ownership of my research. I honestly could not have asked for better supervisors.


\end{acknowledgements}
