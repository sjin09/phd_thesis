% ************************** Thesis Abstract *****************************
% Use `abstract' as an option in the document class to print only the titlepage and the abstract.
\begin{abstract}

Pacific Biosciences' circular consensus sequencing utilises an engineered DNA polymerase with high processivity to sequence the forward and reverse strands of a circular template repeatedly. By leveraging redundancies and complementary base pairing between the two strands, a consensus sequence algorithm generates a highly accurate circular consensus sequence (CCS) read with a minimum read accuracy of Q20.

In this PhD thesis, I propose a hypothesis based on the similarities between duplex and CCS library preparation protocols. I conjecture that a CCS read is equally or more accurate than a duplex read, commonly used for ultra-rare somatic mutation detection. I design and develop himut, a bioinformatics software that leverages the read length and high base accuracy of CCS reads for detecting and phasing somatic mutations in bulk normal tissue, agnostic of species and clonality. 

To demonstrate that a subset of CCS bases has sufficiently high base accuracy for single-molecule somatic mutation, I select a set of positive control samples with characterised somatic mutational processes and a negative control sample to establish the limit of detection. Mutational spectrum derived from somatic mutations were consistent with the expected mutational spectrum from the positive control samples, showing that single-molecule somatic mutation detection is possible with CCS reads. Additionally, using cord blood granulocytes with few somatic mutations, I empirically calculate Q93 CCS base accuracy to range from Q60 to Q90, depending on the substitution error and trinucleotide sequence context, thereby defining the limit of detection using CCS reads. Furthermore, I show that false positives are the result of inaccurate base quality score estimation during CCS generation, and not an inherent property of CCS library preparation and sequencing. 

Afterwards, I use himut to detect somatic mutations and discover new mutational processes from various eukaryotic species sequenced and assembled through the Darwin Tree of Life (DToL) project, which aims to sequence and generate reference genomes for approximately 70,000 eukaryotic species in Great Britain and Ireland. At the time of writing (April 2024), the DToL consortium has publicly released reference genomes for approximately 1600 eukaryotic species. The phylogenetic analysis of new germline and somatic mutational processes identified mutational processes conserved across various levels of taxonomic classification. An example of this is the conservation of C>T somatic mutations at NCG trinucleotides, which result from the spontaneous deamination of 5-methylcytosine to thymine, in Animal, Fungi and Plant Kingdoms.
 
I believe that the methods and the analysis described in this PhD thesis will facilitate the discovery and analysis of new somatic mutational processes across all forms of life. 
  
\end{abstract}
