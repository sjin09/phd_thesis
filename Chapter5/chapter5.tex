%!TEX root = ../thesis.tex
%*******************************************************************************
%****************************** Third Chapter **********************************
%*******************************************************************************
\chapter{Conclusion and Discussion}

% **************************** Define Graphics Path **************************
\ifpdf
    \graphicspath{{Chapter5/Figs/Raster/}{Chapter5/Figs/PDF/}{Chapter5/Figs/}}
\else
    \graphicspath{{Chapter5/Figs/Vector/}{Chapter5/Figs/}}
\fi

%% history of sequencing
%% sanger sequencing
%% cheapter Illumina sequencing
%% solexa
%% rolling circle amplification based approaches
%% Oxford Nanopore Technologies (legal dispute between ONT and PacBio)
%% the advent of PacBio CCS sequencing
%% the limitations of PacBio CCS sequencing
%% the advantages of PacBio CCS sequencing
%% alternative methods towards single molecule somatic mutation
%% PacBio CCS reads are more accurate than duplex reads, but less accurate than nanorate reads
%% PacBio CCS sequencing could increase throughput and thereby lower the cost per base sequencing by increasing the read-of-insert length and increasing the number of ZMWs per SMRTcell.
%% PacBio have approached this in the past and there is no reason why it should not happen
%% Similar to transisoter trechnology per CPU chip
%% whole-genome CCS sequenncing allows users to perform de novo assembly, 5mC detection, somatic mutation detection, germline mutation detection from a single run, providing the most comprehensive set of both genetic and epigenetic information to scientsits
%% to obtain similiar set of depth and breadth of information using Illumina sequencing would cost more and provide data that has less resolution.
%% in addition, obtaining some information requires arduous laboratory procedures or modified library protocols to increase the base quality scores
%% the deceleration in the cost of sequencing from Illumina as Illumina dominated the sequencing market
%% challengers: ONT, PACB, BGI, another company ... cant' remember
%% the longer reads from third-generation sequencing platforms allows, despite the average lower base accuracy, confident placement of reads relative to the reference genome
%% infinity from illumina is based on clever library prep, but it is not a true single molecule sequencing
%% higher base accuracy allows lower sequence coverage to call germline mutations as less evidence is required to have equal confidence in calling germline mutations
%% similarily, if the base accuracy is sufficiently high that sequencign errors can be distinguished from somatic mutations, a mismatch between a single read and the reference genome is a true mutation instead of a sequencing error unless that mismatch was created through DNA damage during library preparation or incorrect repair of DNA damage during library preparation.

\section{Conclusion}
In this PhD thesis, we have discussed the advantages and disadvantages of PacBio SMRT sequencing platform. Before the introduction of circular consensus sequencing, PacBio optimised for read length instead of base accuracy and offered continuous long read sequencing with average read length between 5kb and 20kb and error rate of 10-15\%. CLR reads, hence, were limited to de novo assembly and germline structural variation detection. The advent of CCS reads, however, is a instrumental/monumental momenet in human genomics on multiple-levels. We never had a readout of genetic sequences at this accuracy at this scale with this level of base accuracy. CCS reads have an average read accuracy of Q20 and above, but CCS reads have base accuracy between 1 and 93 with a nominal error rate of 1 error per 5 billion bases. To date, there has not been an independent assessment of PacBio CCS base accuracy except for data described in this PhD thesis. We estimate the empirical error rate of Q93 CCS bases to be between Q60 and Q95 and the error rate is dependent on the substitution and the trinucleotide sequence context. In addition, PacBio has informed us that they use a dinucleotide sequence context hidden markov model for consensus sequence generation and base accuracy estimation, and the limited observation of sequence context might be responsible for the erroneous base accuracy estimation. Moreover, we were able to recover mutational pattern that was more consistent with the gold-standard mutational pattern from the sample when we recalibrated the base quality scores, providing further evidence that base quality scores are erroneously calculated for each base for each trinucleotide sequence context. It is unclear whether how the erroneous bases are introduced to the CCS reads and these erroneous bases must be introduced upstream of the sequencing process or be a result of systematic sequencing error, but a better consensus sequence algorithm will be able to address this problem in the future. We, furthermore, observed that somatic mutations called from shorter CCS reads have a higher number of false positive mutations than that called from longer CCS reads. Our hypothesis is that template with read-of-insert will have higher number of full passes and hence, more bases will be assigned Q93 base quality score, increasing the likelihood that erroenous library errors are assigned a high base quality score. In addition, we have observed in one of our sperm samples and in some of the DToL samples where Blue Pippin based size selection prior to CCS library preparation will introduce DNA damage to the template DNA such that C>T mutations are elevated in the overall mutation call. For a damage introduced upstream of CCS library preparation to have Q93, the DNA damage must be repaired such that the DNA base on both the forward and reverse strand is erroneously repaired. We hypothesied that ** might be responsible for this type of erroenous DNA damage repair.  Hence, a combination of library errors and consensus sequencing errors are present currently in the CCS reads. Since himut relies on base quality score as one of the features of single molecule somatic mutation calling, the increase in the proportion of bases with Q93 bases leads to distortions in the number of absolute number of called mutations and decreases sensitivity.

In comparison to the traditional next-generation sequencing methods, CCS reads have longer read length, is free from PCR amplification and has higher base accuracy. Despite these limitations, PacBio CCS reads outperform on every metric from read length, base accuracy, number of applications from a single run compared to short reads from next-generation sequencing except for per base sequencing cost. This, however, is a limitation that PacBio as a company can overcome through a number of ways: i) the number of ZMWs per SMRTcell can be increased and ii) the average read-of-insert length can be incresed per template molecule. PacBio has increased the number of ZMWS per SMRTcell from XX ZMWs in XXXX to 8 million ZMWs to XXXX. In addition, the average read-of-insert length for CCS sequencing has increased from 10kb in 2019 to 20kb to 2021. Morever, if PacBio is further able to increase the processivitiy of DNA polymerase through further protein engineering or DNA polymerase evolution, they will be able to choose between longer average read-of-insert lnegth or increase in base accuracy through increases in the number of passes per template. I would assume that PacBio will choose to increase the read-of-insert length instead of base accuracy as base accuracy is certaintly sufficiently high at the moment for most practical purposes and higher than what is offered through NGS platforms. In addition, our research suggests that PacBio CCS base accuracy problem should be resolved not through increase in the number of passes per read, but through better design of their conensus sequence algorithm. Recently, Google released deepConsensus algorithm to polish CCS reads based on alignment of subreads from the same ZMW to the CCS reads and to recalibrate the base quality scores. Deepconsensus, currently, cannot be applied towards all the CCS reads produced from SMRTcell and instead must be applied a subset of CCS reads for an average user. In addition, deepConsensus fails to estimate the base accuracy of the reads properly and the base accuracy estimates are too pessimistic, ranging from Q1 to Q50, which is below our empirical estimate between Q60 and Q90 for Q93 bases. In addition, if somatic mutations are called from CCS reads with polished with deepConsensus using Q50 bases, we are not able to obtain a mutational pattern that is expected from the sample.


Based on our understanding of CCS characteristics, we attempted to search for genomic events that could not be captured with short read sequencing and that could, however, be captured PacBio CCS sequencing. We hypotheised that PacBio CCS reads will also have sufficient base accuracy to detect gene conversions and crossovers from both sperm during meiotic recombniation, granulocytes from Bloom syndrome patients and normal individuals during mitotic recombination. Gene conversion and crossover detection necessitates haplotype phasing of multiple kilobases and detection of haplotype rearrangement that might occur in a single sperm or a single cell. 


Despite these limitations, as HMW DNA input requirements for CCS library preparation decreaess and as sequence throughput and sequencing cost decreases, I believe that PacBio CCS sequencing might be the last DNA sequencing platform to dominate the sequencing market.









